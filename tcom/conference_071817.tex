\documentclass[conference]{IEEEtran}
\IEEEoverridecommandlockouts
% The preceding line is only needed to identify funding in the first footnote. If that is unneeded, please comment it out.
\usepackage{cite}
\usepackage{amsmath,amssymb,amsfonts}
% \usepackage{algorithmic}
\usepackage{graphicx}
\usepackage{textcomp}
\def\BibTeX{{\rm B\kern-.05em{\sc i\kern-.025em b}\kern-.08em
    T\kern-.1667em\lower.7ex\hbox{E}\kern-.125emX}}
\begin{document}

\title{Image Steganography\\
}

\author{\IEEEauthorblockN{Saketh Katari}
\IEEEauthorblockA{\textit{Computer Science Department} \\
\textit{IIIT-Delhi}\\
Delhi, India \\
katari15045@iiitd.ac.in}
}

\maketitle

\begin{abstract}
This document explains a new strategy to transfer confidential data by combining steganography with cryptography.
LSB steganography is further improved by randomly storing confidential data in an image which makes it difficult to retrieve.
Considering human visual system, it is difficult for the user to detect the changes in stego image. \\
\end{abstract} 

\begin{IEEEkeywords}
steganography, LSB, image, cryptography, random
\end{IEEEkeywords}

\section{Introduction}
Confidential data is everywhere and it is necessary to transfer it securely over internet.
Cryptography comes into picture when the data is with you but nothing is made out of it, as it is a cipher, making no sense.
Steganography hides confidential data within some other data (say image) and makes it difficult to even detect that there is a secret here.
Cryptography combined with steganography provides a 2 layer protection - detecting the confidential data (steganography) and making sense out of it (Cryptography).\\

Images are used (among audio, video, text etc) as a medium to store the confidential data.
Any data such as text, image, audio etc can be embedded in an image and is tranferred to the destination.
If an attacker sees this image, it would be difficult to detect that there is some hidden data in it, as it looks lika a normal image.\\

\section{Related work}
Most of the traditional LSB steganography techniques store the confidential data in the LSBs (Least Significant Bit) of every pixel without any randomization. 
This is not secure because, an attacker might just access the LSBs of every pixel and retrieve the confidential data. 
Instead, some randomization is necessary while embedding data so that, when an attacker tries to retrieve data, it is difficult to produce the same pattern that was randomly generated.\\

However, it is necessary to transfer the random pattern to the destination to retrieve data that is embedded. \\

During the process of embedding data in an image, the LSBs of some pixels are modified which modifies the image.
As the LSB of any binary number corresponds to 0 or 1, it has very less effect (changes by 1) on the overall value.\\ 

The performance of a method or an algorithm is evaluated on the basis of the change in the image before and after embedding data in it.
An algorithm is considered better if it produces minimal change in the image after embedding data in it.\\

\section{Proposed model}
This model adds cryptography and randomization to the traditional LSB steganographic techniques.

\subsection{Notation}
Cover\_image - The image (before embedding) that is supposed to carry the confidential data in it. \\

Stego\_image - The image (after embedding) that contains confidential data in it. \\

location\_data - Meta data (Data about confidential\_data) that identifies the location of confidential\_data within a stego\_image. \\

Component\_of\_a\_pixel - The Red, Green and the Blue components (RGB). \\

LSB - Least Significant Bit of a binary number. \\

pr\_key\_src - Sender's private key \\

pr\_key\_dest - Receiver's private key \\

pub\_key\_src - Sender's public key \\

pub\_key\_dest - Receiver's public key \\

\subsection{Pre-processing}
A random location\_data is generated with necessary length. Let it be 'abc'.
This is converted to 8-bit ASCII values '97, 98, 99'.
Finally, it is converted to a binary string '01100001, 01100010, 01100011'.\\

Similarly, another binary string is produced for the confidential\_data.\\

\subsection{Embedding}
Each bit is taken from the confidential\_data and from the location\_data.
Let them be c\_bit and l\_bit respectively.
c\_bit is embedded in one of the components of a pixel which is determined by l\_bit.\\

It is observerd that the human eye is most sensitive to the green component of the light.
So, LSBs of green component are not replaced.\\

A pixel is chosen in lexicographical order (row\_number, column\_number in lexicographical order).
For a given l\_bit and c\_bit, l\_bit is XORed with LSB of green component.
If it gives 0, LSB of blue is replaced with c\_bit, else, LSB of red is replaced with c\_bit.\\

\subsection{Encryption}
Stego\_image is first encrypted by the pr\_key\_src and then encrypted with the pub\_key\_dest.
pr\_key\_src is used as a digital signature.
pub\_key\_dest makes sure that only destination can decrypt it with it's private key.\\

Similarly, location\_data is encrypted.\\

Now, stego\_image along with location\_data is transferred to destination.

\subsection{Decrpytion}
Stego\_image is first decrypted using the pr\_key\_dest and then with the pub\_key\_src.
Now, the receiver has the decrypted stego\_image .\\

Similarly, decrypt the location\_data. \\
\subsection{Extracting embedded data}
A pixel is chosen in lexicographical order (row\_number, column\_number in lexicographical order).
For each bit l\_bit in location\_data, XOR l\_bit with the LSB of the green component.
If it's 0, append the LSB of blue to the confidential\_data, else, append the LSB of red to the confidential\_data.\\

Convert confidential\_data to 8-bit ASCII values and finally to ASCII characters.\\
\subsection{PSNR (Peak Signal to Noise Ratio)}
As name indicates, it is the ratio of signal and noise.
Higher the PSNR lesser the noise in an image.\\

PSNR of traditional LSB based steganography techniques is around 53.
This method gives a PSNR of around 56, better than traditional PSNR.\\  
\subsection{Security}
It has two layers of security.
First layer being steganography, second being cryptography.\\

It's difficult to detect whether there is confidential\_data in an image.
If an attacker somehow gets encrypted data, it's difficult to decrypt it.\\
\begin{table}[htbp]
\caption{Table Type Styles}
\begin{center}
\begin{tabular}{|c|c|c|c|}
\hline
\textbf{Table}&\multicolumn{3}{|c|}{\textbf{Table Column Head}} \\
\cline{2-4} 
\textbf{Head} & \textbf{\textit{Table column subhead}}& \textbf{\textit{Subhead}}& \textbf{\textit{Subhead}} \\
\hline
copy& More table copy$^{\mathrm{a}}$& &  \\
\hline
\multicolumn{4}{l}{$^{\mathrm{a}}$Sample of a Table footnote.}
\end{tabular}
\label{tab1}
\end{center}
\end{table}

Figure Labels: Use 8 point Times New Roman for Figure labels. Use words 
rather than symbols or abbreviations when writing Figure axis labels to 
avoid confusing the reader. As an example, write the quantity 
``Magnetization'', or ``Magnetization, M'', not just ``M''. If including 
units in the label, present them within parentheses. Do not label axes only 
with units. In the example, write ``Magnetization (A/m)'' or ``Magnetization 
\{A[m(1)]\}'', not just ``A/m''. Do not label axes with a ratio of 
quantities and units. For example, write ``Temperature (K)'', not 
``Temperature/K''.

\section*{Acknowledgment}

The preferred spelling of the word ``acknowledgment'' in America is without 
an ``e'' after the ``g''. Avoid the stilted expression ``one of us (R. B. 
G.) thanks $\ldots$''. Instead, try ``R. B. G. thanks$\ldots$''. Put sponsor 
acknowledgments in the unnumbered footnote on the first page.

\section*{References}

Please number citations consecutively within brackets \cite{b1}. The 
sentence punctuation follows the bracket \cite{b2}. Refer simply to the reference 
number, as in \cite{b3}---do not use ``Ref. \cite{b3}'' or ``reference \cite{b3}'' except at 
the beginning of a sentence: ``Reference \cite{b3} was the first $\ldots$''

Number footnotes separately in superscripts. Place the actual footnote at 
the bottom of the column in which it was cited. Do not put footnotes in the 
abstract or reference list. Use letters for table footnotes.

Unless there are six authors or more give all authors' names; do not use 
``et al.''. Papers that have not been published, even if they have been 
submitted for publication, should be cited as ``unpublished'' \cite{b4}. Papers 
that have been accepted for publication should be cited as ``in press'' \cite{b5}. 
Capitalize only the first word in a paper title, except for proper nouns and 
element symbols.

For papers published in translation journals, please give the English 
citation first, followed by the original foreign-language citation \cite{b6}.

\begin{thebibliography}{00}
\bibitem{b1} G. Eason, B. Noble, and I. N. Sneddon, ``On certain integrals of Lipschitz-Hankel type involving products of Bessel functions,'' Phil. Trans. Roy. Soc. London, vol. A247, pp. 529--551, April 1955.
\bibitem{b2} J. Clerk Maxwell, A Treatise on Electricity and Magnetism, 3rd ed., vol. 2. Oxford: Clarendon, 1892, pp.68--73.
\bibitem{b3} I. S. Jacobs and C. P. Bean, ``Fine particles, thin films and exchange anisotropy,'' in Magnetism, vol. III, G. T. Rado and H. Suhl, Eds. New York: Academic, 1963, pp. 271--350.
\bibitem{b4} K. Elissa, ``Title of paper if known,'' unpublished.
\bibitem{b5} R. Nicole, ``Title of paper with only first word capitalized,'' J. Name Stand. Abbrev., in press.
\bibitem{b6} Y. Yorozu, M. Hirano, K. Oka, and Y. Tagawa, ``Electron spectroscopy studies on magneto-optical media and plastic substrate interface,'' IEEE Transl. J. Magn. Japan, vol. 2, pp. 740--741, August 1987 [Digests 9th Annual Conf. Magnetics Japan, p. 301, 1982].
\bibitem{b7} M. Young, The Technical Writer's Handbook. Mill Valley, CA: University Science, 1989.
\end{thebibliography}

\end{document}
